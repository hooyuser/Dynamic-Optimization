\documentclass{article}
% Comment the following line to NOT allow the usage of umlauts
\usepackage[utf8]{inputenc}
% Uncomment the following line to allow the usage of graphics (.png, .jpg)
%\usepackage{graphicx}
\usepackage{amsmath}
% Start the document
\begin{document}

% Create a new 1st level heading
\section{Finite Horizon Optimization}
\textsc{Problem 1}
$$
\max_{\mathbf{a_t},\mathbf{c_t}}\sum_{t=0}^T\beta^t u(c_t) 
$$
$$
s.t.\; a_{t+1}=Ra_t-c_t\;(t=0,1,\cdots ,T),a_{T+1}=0,
$$
where $\mathbf{a_t}=\{a_t\}_{t=1}^{T+1},\mathbf{c_t}=\{c_t\}_{t=0}^{T}$ and $a_0$, $\beta$, $T$ are given constants.\\ 


\noindent\textbf{Solution 1} 
Write down Lagrangian function 
$$
L(\mathbf{a_t},\mathbf{c_t},\lambda)=\sum_{t=0}^T\beta^t[u(c_t)-\lambda_t(a_{t+1}-Ra_t-c_t)]+\lambda_{T+1}a_{T+1}
,
$$
and we get F.O.C. 
$$
\frac{\partial L}{\partial a_t}=\beta^t\lambda_t R-\beta^{t-1}\lambda_{t-1}=0\;(t=1,2,\cdots ,T),
$$


$$
\frac{\partial L}{\partial a_{T+1}}=\lambda_{T+1}-\beta^{T}\lambda_{T}=0,
$$

$$
\frac{\partial L}{\partial c_t}=\beta^t[u'(c_t)-\lambda_t]=0\;(t=0,1,\cdots ,T).
$$
which gives
$$
\beta R=\frac{u'(c_{t-1})}{u'(c_t)}(t=1,2,\cdots ,T)
$$


\noindent\textbf{Solution 2}
The constraint conditions $a_{t+1}=Ra_t-c_t\;(t=0,1,\cdots ,T)$ can be rewritten as
\[
\frac{a_{t+1}}{R^{t+1}}=\frac{a_{t}}{R^{t}}-\frac{c_{t}}{R^{t+1}}\;(t=0,1,\cdots ,T).
\]
Adding up the equations above leads to
\[
\frac{a_{T+1}}{R^{T+1}}=a_{0}-\sum_{t=0}^T\frac{c_{t}}{R^{t+1}}\;(t=0,1,\cdots ,T).
\]
By applying $a_{T+1}=0$ we get the equivalent optimization problem
$$
\max_{\mathbf{c_t}}\sum_{t=0}^T\beta^t u(c_t)
$$
$$
s.t.\; \sum_{t=0}^{T}\frac{c_t}{R^t}=Ra_0,
$$
where $\mathbf{c_t}=\{c_t\}_{t=0}^{T}$. Lagrangian is
$$
L(\mathbf{c_t},\lambda)=\sum_{t=0}^T\beta^t u(c_t)-\lambda\left(\sum_{t=0}^{T}\frac{c_t}{R^t}-Ra_0\right)
,
$$
$$
\frac{\partial L}{\partial c_t}=\beta^t u'(c_t)-\frac{\lambda}{R^t}=0\;(t=0,1,\cdots ,T),
$$
~\\
which gives
$$
\beta R=\frac{u'(c_{t-1})}{u'(c_t)}(t=1,2,\cdots ,T).
$$



\section{Bellman Equation}
\textsc{Problem 2}
$$
\max_{\mathbf{k_t},\mathbf{c_t}}\sum_{t=0}^\infty \beta^t u(c_t) 
$$
$$
s.t.\; k_{t+1}=f(k_t)+k_t-\delta k_t-c_t\;(t=0,1,\cdots),k_{t}\ge 0,
$$
where $\mathbf{k_t}=\{k_t\}_{t=1}^{\infty},\mathbf{c_t}=\{c_t\}_{t=0}^{\infty}$ and $k_0$, $\beta$, $\delta$ are given constants.\\ 

\noindent\textbf{Solution} 
Bellman equation can be stated as
$$
V(k_t)=\max_{c_t}\{u(c_t)+\beta V(k_{t+1})\},
$$
where
\[
V(k_T)=\max_{\mathbf{k_t},\mathbf{c_t}}\sum_{t=T}^\infty \beta^{t-T} u(c_t) 
\]
$$
s.t.\; k_{t+1}=f(k_t)+k_t-\delta k_t-c_t\;(t=T,T+1,\cdots),k_{t}\ge 0.
$$
According to envelope theorem, we have
$$
\frac{\partial V(k_t)}{\partial k_t}=V'(k_t)=\beta V'(k_{t+1})[f'(k_t)+1-\delta],
$$
$$
\frac{\partial V(k_t)}{\partial c_t}=0=u'(c_t)-\beta V'(k_{t+1}).
$$
The system of simultaneous equations derives 
$$
u'(c_t)=\beta u'(c_{t+1})[f'(k_{t+1})+1-\delta ]
$$
Let $$F(k_t)=f(k_t)+k_t-\delta k_t$$ and we can rewrite it as 
\begin{equation}
	u'(F(k_t)-k_{t+1})=\beta u'(F(k_{t+1})-k_{t+2})F'(k_{t+1}).
\end{equation}
So as to find a stable solution $k^*$, we can solve
$$
u'(F(k^*)-k^*)=\beta u'(F(k^*)-k)F'(k^*)
$$
namely
$$
\beta F'(k^*)=1.
$$
When $t$ is sufficiently large, $k_t$ will be sufficiently close to $k^*$ while $c_t$ approaches $c^*$. Let
$$
g(k_t,k_{t+1},k_{t+2})=-u'(F(k_t)-k_{t+1})+\beta u'(F(k_{t+1})-k_{t+2})F'(k_{t+1}).
$$
We see $g(k_t,k_{t+1},k_{t+2})=0$ at $(k_t,k_{t+1},k_{t+2})=(k^*,k^*,k^*)$. Take Taylor expansion of order one at $\mathbf{k^*}=(k^*,k^*,k^*)$ and we get the linearization of $g$
\begin{align*}	
&  \widetilde{g}(k_t,k_{t+1},k_{t+2})\\
= &\;g(\mathbf{k^*})+\frac{\partial g(\mathbf{k^*})}{\partial k_t}(k_t-k^*)+\frac{\partial g(\mathbf{k^*})}{\partial k_{t+1}}(k_{t+1}-k^*)+\frac{\partial g(\mathbf{k^*})}{\partial k_{t+2}}(k_{t+2}-k^*)\\
= & -u''(c^*)F'(k^*)(k_t-k^*)+[u''(c^*)+\beta u''(c^*)F'(k^*)^2+\beta u'(c^*)F''(k^*)](k_{t+1}-k^*)\\
& -\beta u''(c^*)F'(k^*)(k_{t+2}-k^*).
\end{align*}
Indeed, we have also linearized the nonlinear implicit difference equation (1). Let $z_t=k_t-k^*$. The following second order difference equation can describe the behavior of $k_t$ approximately when $t$ is sufficiently large
\[
z_{t+2}= \left[\frac{1}{\beta F'(k^*)}+ F'(k^*)+\frac{ u'(c^*)/u''(c^*)}{F'(k^*)/F''(k^*)}\right]z_{t+1}-\frac{1}{\beta}z_t
\]
Since $\beta F'(k^*)=1$, let $s=\dfrac{ u'(c^*)/u''(c^*)}{F'(k^*)/F''(k^*)}$ and we can give the matrix form of the difference equation
$$
\begin{bmatrix}
z_{t+2}\\
z_{t+1}  
\end{bmatrix}
=
\begin{bmatrix}
1+\beta^{-1}+s&-\beta^{-1}\\
1  &0
\end{bmatrix}
\begin{bmatrix}
z_{t+1}\\
z_{t}  
\end{bmatrix}
=A
\begin{bmatrix}
z_{t+1}\\
z_{t}  
\end{bmatrix}.
$$
The characteristic polynomial of $A$ is
\[
p(\lambda)=|\lambda E-A|=\lambda^2-(1+\beta^{-1}+s)\lambda+\beta^{-1}.
\]
Note $p(0)=\beta^{-1}>0$ and $p(1)=-s<0$. It implies $A$ has two positive real roots $\lambda_1< 1$ and $\lambda_2>1$. Let 
$$
A=TJT^{-1}=
\begin{bmatrix}
\beta^{-1}&\lambda_2\\
\lambda_1+1+\beta^{-1}+s&1  
\end{bmatrix}
\begin{bmatrix}
\lambda_1&\\
 &\lambda_2
\end{bmatrix}
\begin{bmatrix}
\beta^{-1}&\lambda_2\\
\lambda_1+1+\beta^{-1}+s&1  
\end{bmatrix}^{-1}
.
$$
We have 
\[
A^n=T
\begin{bmatrix}
\lambda_1^n&\\
&\lambda_2^n
\end{bmatrix}
T^{-1}
\]
and
\[
\begin{bmatrix}
z_{t}\\
z_{t-1}  
\end{bmatrix}
=T
\begin{bmatrix}
\lambda_1^{n-1}&\\
&\lambda_2^{n-1}
\end{bmatrix}
T^{-1}
\begin{bmatrix}
z_{1}\\
z_{0}  
\end{bmatrix}
\]

% Uncomment the following two lines if you want to have a bibliography
%\bibliographystyle{alpha}
%\bibliography{document}

\end{document}
